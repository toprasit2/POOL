\section{ทฤษฏีและหลักการ}
		
		การแสดงผลจะเป็นมิตรกับผู้ใช้งานหรือไม่ ผู้ใช้งานจะสามารถใช้งานระบบได้ง่ายหรือไม่  ขึ้นอยู่กับการออกแบบส่วนแสดงผลของระบบโดยผู้พัฒนาจะนำทฤษฏีและหลักการต่อไปนี้มาใช้เพื่อออกแบบและพัฒนาส่วนติดต่อผู้ใช้งานระบบเฝ้าระวังด้วยระบบกล้องวงจรปิดขนาดใหญ่ให้มีประสิทธิภาพและเหมาะสมกับผู้ใช้งาน

		Usability หมายถึง ความสามารถในการเรียนรู้การใช้งานสิ่งที่มนุษย์สร้างขึ้นมาไม่ว่าจะเป็นเครื่องมือหรืออุปกรณ์และสามารถใช้งานเครื่องมือหรืออุปกรณ์นั้นๆได้อย่างง่ายดายและเป็นเกณฑ์ที่จะพิสูจน์ว่าเครื่องมือหรืออุปกรณ์ที่จะสร้างขึ้นนั้นสามารถถูกใช้งานโดยผู้ใช้งานอย่างมีคุณภาพ คุณภาพในที่นี่คือ ประโยชน์ที่ผู้ใช้งานจะได้รับ, สมรรถภาพการใช้งานเครื่องมือหรืออุปกรณ์หรือการที่ผู้ใช้งานสามารถเรียนรู้การใช้งานได้รวมไปถึงความเร็วในการใช้งานให้บรรลุวัตถุประสงค์ และ ความพึงพอใจของผู้ใช้งานในการใช้งานระบบ
	

		User Center Design หมายถึง การออกแบบและวิเคราะห์ระบบโดยยึดถือผู้ใช้งานเป็นหลัก โดยต้องวิเคราะห์สิ่งเหล่านี้
1. ใครเป็นผู้ใช้งานของระบบ
2. งานหรือจุดประสงค์ของผู้ใช้งานคืออะไร
3. ผู้ใช้งานมีประสบการณ์ในการใช้งานระบบที่เกี่ยวของกับระบบที่ออกแบบมากน้อยเพียงใด
3. ผู้ใช้งานมีฟังก์ชันการใช้งานใดบ้างในระบบ
4. ผู้ใช้งานต้องการได้รับข้อมูลอะไรจากระบบ
5. ผู้ใช้งานอยากให้ระบบเป็นอย่างไร
6. มีสิ่งใดบ้างที่เป็นอุปสรรคของระบบ
	การออกแบบแบบ User Center Design จะทำให้ซอฟต์แวร์ที่จะพัฒนามี Usability ซึ่งบ่งบอกว่าซอฟต์แวร์ที่จะพัฒนามีคุณภาพ


		สีเป็นองค์ประกอบหลักสำหรับการตกแต่ง จึงจำเป็นอย่างยิ่งที่จะต้องทำความเข้าใจเกี่ยวกับการใช้สี โดยระบบสีที่แสดงบนจอคอมพิวเตอร์ มีระบบการแสดงผลผ่านหลอดลำแสงที่เรียกว่า Cathode ray tube (CRT) โดยมีลักษณะระบบสีแบบบวก อาศัยการผสมของของแสงสีแดง สีเขียว และสีน้ำเงิน หรือระบบสี RGB นั่นเองซึ่งการรวมสีของแม่สีทั้งสามสีจะทำให้เกิดแสงสีขาวซึ่งต่างจากการผสมสีที่เราเห็นในชีวิตประจำวัน สีแต่ละสีจะส่งผลต่อความรู้สึกของมนุษย์แตกต่างกันออกไป สีบางสีทำให้คนเรารู้สึกสดชื่น ปลอดโปร่งหรือสบายใจและในทางกลับกันก็มีสีที่ทำให้เกิดความรู้สึกหดหู่ กังวลใจหรือเครียดได้เช่นกัน การเลือกใช้สีให้เหมาะสม กลมกลืน ไม่เพียงแต่จะสร้างความพึงพอใจให้กับผู้ใช้งาน แต่ยังสามารถทำให้เห็นถึงความแตกต่างระหว่างเว็บไซต์ได้ โดยแต่ละสีจะสื่อถึงอารมณ์ต่างๆ ดังนี้

		สีฟ้าหรือน้ำเงิน ให้ความรู้สึกสงบ สุขุม สุภาพ หนักแน่น เคร่งขรึม เอาการเอางาน ละเอียด รอบคอบ สง่างาม มีศักดิ์ศรี สูงศักดิ์ เป็นระเบียบถ่อมตน สามารถลดความตื่นเต้นและช่วยทำให้มีสมาธิ แต่ถ้าเป็นสีน้ำเงินเข้มไปจะมำให้รู้สึกซึมเศร้าได้

		สีแดง เป็นสีที่สร้างความตื่นเต้นและกระตุ้นสมอง สีแดงปานกลางแสดงถึงความมีสุขภาพดีความมีชีวิต ความรัก ความสำคัญ ความอุดมสมบูรณ์ ความมั่งคั่ง และนอกจากนี้สีแดงยังมีความหมายแฝงด้านกามารมณ์ การสร้างความรู้สึกรุนแรง เร่าร้อน ท้าทาย ความตื่นเต้น เร้าใจ มีพลัง สีแดงที่จัดมากๆ มีผลรบกวนสายตาและทำให้สายตาเมื่อยล้าได้ง่ายจึงไม่เหมาะนำมาใช้ในงานที่มีการรับชมนานๆ
	